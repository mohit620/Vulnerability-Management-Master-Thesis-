\chapter{Literature Review}
\label{chap:kapitel3}
\section{what is Literature Review}
The authors S. A. Merrell and J. F. Stevens \cite{merrell2008improving} suggests that an Organization should follow a process metrics. An organization can identify areas where it performs poorly, prioritize those areas based on organizational drivers and goals, and then commit resources to improving those areas. This is especially beneficial to an organization's audit function, as these performance measures are precisely the types of data that governance and oversight authorities should be looking at to determine the quality of organizational activities. Frequently, measurement programs focus on the process outputs rather than the process quality. 

To give in depth information of performance matrix and process improvement an example was taken into the picture. Imagine a company that manages software vulnerabilities with the help of regularly used tools (i.e.,the Nessus vulnerability scanner, Defense Information SystemsAgency [DISA] scanning tool, etc.). The programs look at existing software programs, compare their configurations to databases of common vulnerabilities, and provide a "severity grade" to any vulnerabilities found, such as "low," "medium," or “critical." However, most It security operations conflate severity ratings with risk, therefore the reaction to a vulnerability's discovery is predicated on just that grade. Whenever a vulnerability's severity is exclusively measured in terms of a single component, the group's reaction is unlikely to reflect the true danger to the organization. The reliance on tool output prevents security employees from considering the system's operating environment as well as the system's and information's security classification. 

Simply mapping identified vulnerabilities to both impact on the system in which the vulnerability occurs and on the enterprise as a whole is one method to turn this behavior into a more meaningful analysis. This mapping exercise can help to contextualize the impact of vulnerabilities on the enterprise and ensures that recommended fixes can be prioritized based on organizational risk. The organization can use identification and analysis to evaluate whether they need to stop everything and remedy the vulnerability right once, or if updates/patches can wait and go through the organization's usual test and update cycle.
Companies can, preferably, evaluate whether such a given vulnerability needs a repair and resolution or acceptance and monitoring through analysis guided by risk management activities. For example, only if low-priority  assets are at danger, or if the possibility of an exploitation is very distant, the business may be able to tolerate a specific vulnerability. Similarly, if no other security solution is obtained and repair is not cost-effective, the vulnerability may be acknowledged and monitored. Consider the manager who, when a technical risk rated "critical" was discovered in his organization, hung a red flag on his door. They could allow their employees to redirect their attention to more productive projects if they used this type of analysis. 
Conclusion : When vulnerability management activities are linked to corporate information security risk management, a group's vulnerability management activities improve. They'll be able to contextualize threat and vulnerability information inside enterprise risk, and make better decisions about how to allocate limited resources to prepare for, detect, and respond to vulnerabilities.

			