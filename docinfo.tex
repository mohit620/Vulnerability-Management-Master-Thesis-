% -------------------------------------------------------
% Daten für die Arbeit
% Wenn hier alles korrekt eingetragen wurde, wird das Titelblatt
% automatisch generiert. D.h. die Datei titelblatt.tex muss nicht mehr
% angepasst werden.

\newcommand{\hsmasprache}{en} % de oder en für Deutsch oder Englisch
% Für korrekt sortierte Literatureinträge, noch preambel.tex anpassen
% und zwar bei \usepackage[main=ngerman, english]{babel},
% \usepackage[phttps://www.overleaf.com/projectagebackref=false,german]{hyperref}
% und \usepackage[autostyle=true,german=quotes]{csquotes}

% Titel der Arbeit auf Deutsch
\newcommand{\hsmatitelde}{Methodology For Improving Vulnerability Management}

% Titel der Arbeit auf Englisch
\newcommand{\hsmatitelen}{Methodology For Improving Vulnerability Management}

% Weitere Informationen zur Arbeit
\newcommand{\hsmaort}{Offenburg}    % Ort
\newcommand{\hsmaautorvname}{Mohit Lakhichand Patil} % Vorname(n)
\newcommand{\hsmaautornname}{Fernando} % Nachname(n)
\newcommand{\hsmadatum}{12.04.2022} % Datum der Abgabe
\newcommand{\hsmajahr}{2022} % Jahr der Abgabe
%\newcommand{\hsmafirma}{XYZ} % Firma bei der die Arbeit durchgeführt wurde
\newcommand{\hsmabetreuer}{Prof Dr rer soc (HSG) Dirk
Drechsler, Hochschule Offenburg} % Betreuer an der Hochschule
%\newcommand{\hsmazweitkorrektor}{Martina Musterfrau, Unternehmen} % Betreuer im Unternehmen oder Zweitkorrektor
\newcommand{\hsmafakultaet}{EMI} % Fakultät
\newcommand{\hsmastudiengang}{ENITS} % Studiengangsabkürzung. 
% Diese wird in titelblatt.tex definiert. Bisher AI, EI, MK und INFM. Bitte ergänzen.

% Zustimmung zur Veröffentlichung
\setboolean{hsmapublizieren}{true}   % Einer Veröffentlichung wird zugestimmt
\setboolean{hsmasperrvermerk}{false} % Die Arbeit hat keinen Sperrvermerk

% -------------------------------------------------------
% Abstract

% Kurze (maximal halbseitige) Beschreibung, worum es in der Arbeit geht auf Deutsch
\newcommand{\hsmaabstractde}{Lorem ipsum dolor sit amet, consetetur sadipscing elitr, sed diam nonumy eirmod tempor invidunt ut labore et dolore magna aliquyam erat, sed diam voluptua. At vero eos et accusam et justo duo dolores et ea rebum. Stet clita kasd gubergren, no sea takimata sanctus est Lorem ipsum dolor sit amet. Lorem ipsum dolor sit amet, consetetur sadipscing elitr, sed diam nonumy eirmod tempor invidunt ut labore et dolore magna aliquyam erat, sed diam voluptua. At vero eos et accusam et justo duo dolores et ea rebum. Stet clita kasd gubergren, no sea takimata sanctus est Lorem ipsum dolor sit amet.}

% Kurze (maximal halbseitige) Beschreibung, worum es in der Arbeit geht auf Englisch

\newcommand{\hsmaabstracten}{Even though the internet has only been there for a short period, it has grown tremendously. Today, a significant portion of commerce is conducted entirely online because of increased internet users and technological advancements in web construction. Additionally, cyberattacks and threats have expanded significantly, leading to financial losses, privacy breaches, identity theft, a decrease in customers' confidence in online banking and e-commerce, and a decrease in brand reputation and trust. When an attacker pretends to be a genuine and trustworthy institution, they can steal private and confidential information from a victim. Aside from that, phishing has been an ongoing issue for a long time. Billions of dollars have been shed on the global economy.
In recent years, there has been significant progress in the development of phishing detection and identification systems to protect against phishing attacks. Phishing detection technologies frequently produce binary results, i.e., whether a phishing attempt was made or not, with no explanation. On the other hand, phishing identification methodologies identify phishing webpages by visually comparing webpages with predetermined authentic references and reporting phishing together with its target brand, resulting in findings that are understandable. However, technical difficulties in the field of visual analysis limit the applicability of currently available solutions, preventing them from being both effective (with high accuracy) and efficient (with little runtime overhead).
Here, we evaluate existed framework called  Phishpedia. This hybrid deep learning system can recognize identity logos from webpage screenshots and match logo variants of the same brand with high precision. Phishpedia provides high accuracy with low runtime. Lastly, unlike other methods, Phishpedia does not require training on any phishing samples whatsoever. Phishpedia exceeds baseline identification techniques (EMD, PhishZoo, and LogoSENSE), inaccurately detecting phishing pages in lengthy testing using accurate phishing data. The effectiveness of Phishpedia was tested and compared against other standard machine learning algorithms and some state-of-the-art algorithms. The given solutions performed better than different algorithms in the given dataset, which is impressive.}
